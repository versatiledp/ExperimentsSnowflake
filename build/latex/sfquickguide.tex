%% Generated by Sphinx.
\def\sphinxdocclass{report}
\documentclass[letterpaper,10pt,english]{sphinxmanual}
\ifdefined\pdfpxdimen
   \let\sphinxpxdimen\pdfpxdimen\else\newdimen\sphinxpxdimen
\fi \sphinxpxdimen=.75bp\relax

\PassOptionsToPackage{warn}{textcomp}
\usepackage[utf8]{inputenc}
\ifdefined\DeclareUnicodeCharacter
% support both utf8 and utf8x syntaxes
  \ifdefined\DeclareUnicodeCharacterAsOptional
    \def\sphinxDUC#1{\DeclareUnicodeCharacter{"#1}}
  \else
    \let\sphinxDUC\DeclareUnicodeCharacter
  \fi
  \sphinxDUC{00A0}{\nobreakspace}
  \sphinxDUC{2500}{\sphinxunichar{2500}}
  \sphinxDUC{2502}{\sphinxunichar{2502}}
  \sphinxDUC{2514}{\sphinxunichar{2514}}
  \sphinxDUC{251C}{\sphinxunichar{251C}}
  \sphinxDUC{2572}{\textbackslash}
\fi
\usepackage{cmap}
\usepackage[T1]{fontenc}
\usepackage{amsmath,amssymb,amstext}
\usepackage{babel}



\usepackage{times}
\expandafter\ifx\csname T@LGR\endcsname\relax
\else
% LGR was declared as font encoding
  \substitutefont{LGR}{\rmdefault}{cmr}
  \substitutefont{LGR}{\sfdefault}{cmss}
  \substitutefont{LGR}{\ttdefault}{cmtt}
\fi
\expandafter\ifx\csname T@X2\endcsname\relax
  \expandafter\ifx\csname T@T2A\endcsname\relax
  \else
  % T2A was declared as font encoding
    \substitutefont{T2A}{\rmdefault}{cmr}
    \substitutefont{T2A}{\sfdefault}{cmss}
    \substitutefont{T2A}{\ttdefault}{cmtt}
  \fi
\else
% X2 was declared as font encoding
  \substitutefont{X2}{\rmdefault}{cmr}
  \substitutefont{X2}{\sfdefault}{cmss}
  \substitutefont{X2}{\ttdefault}{cmtt}
\fi


\usepackage[Bjarne]{fncychap}
\usepackage{sphinx}

\fvset{fontsize=\small}
\usepackage{geometry}


% Include hyperref last.
\usepackage{hyperref}
% Fix anchor placement for figures with captions.
\usepackage{hypcap}% it must be loaded after hyperref.
% Set up styles of URL: it should be placed after hyperref.
\urlstyle{same}


\usepackage{sphinxmessages}
\setcounter{tocdepth}{2}



\title{sfquickguide}
\date{Sep 20, 2020}
\release{9/28/2020}
\author{Raj Rathi}
\newcommand{\sphinxlogo}{\vbox{}}
\renewcommand{\releasename}{Release}
\makeindex
\begin{document}

\pagestyle{empty}
\sphinxmaketitle
\pagestyle{plain}
\sphinxtableofcontents
\pagestyle{normal}
\phantomsection\label{\detokenize{index::doc}}



\chapter{Main titles}
\label{\detokenize{in_file:main-titles}}\label{\detokenize{in_file::doc}}
Note that there must be as many equals signs as title characters.


\section{Custom stylesheet}
\label{\detokenize{in_file:custom-stylesheet}}\label{\detokenize{in_file:id1}}
Hello here para 01 testing gooing on to check TOC


\section{Title are underlined}
\label{\detokenize{in_file:title-are-underlined}}
Hello here para 02 testing gooing on to check TOC


\subsection{Subtitles with dashes}
\label{\detokenize{in_file:subtitles-with-dashes}}
You can put text in \sphinxstyleemphasis{italic} or in \sphinxstylestrong{bold}, you can “mark” text as code with double backquote \sphinxcode{\sphinxupquote{print()}}.

Special characters can be escaped using a backslash, e.g. \textbackslash{} or *.

Lists are similar to Markdown, but a little more involved.

Remember to line up list symbols (like \sphinxhyphen{} or *) with the left edge of the previous text block, and remember to use blank lines to separate new lists from parent lists:
\begin{itemize}
\item {} 
First item

\item {} 
Second item
\begin{itemize}
\item {} 
Sub item

\end{itemize}

\item {} 
Third item

\end{itemize}

or
\begin{itemize}
\item {} 
First item

\item {} 
Second item
\begin{itemize}
\item {} 
Sub item

\end{itemize}

\item {} 
Third item

\end{itemize}

Tables are really easy to write:


\begin{savenotes}\sphinxattablestart
\centering
\begin{tabulary}{\linewidth}[t]{|T|T|}
\hline
\sphinxstyletheadfamily 
Country
&\sphinxstyletheadfamily 
Capital
\\
\hline
France
&
Paris
\\
\hline
Japan
&
Tokyo
\\
\hline
\end{tabulary}
\par
\sphinxattableend\end{savenotes}

More complex tables can be done easily (merged columns and/or rows) but I suggest you to read the complete doc for this :)

There are multiple ways to make links:
\begin{itemize}
\item {} 
By adding an underscore after a word : \sphinxhref{https://github.com/}{Github} and by adding the target URL after the text (this way has the advantage to not insert unnecessary URLs inside readable text).

\item {} 
By typing a full comprehensible URL : \sphinxurl{https://github.com/} (will be automatically converted to a link)

\item {} 
By making a more Markdown\sphinxhyphen{}like link: \sphinxhref{https://github.com/}{Github} .

\end{itemize}

\begin{DUlineblock}{0em}
\item[] Hello Plunk How you do with all this nonsense of writing around.. I :red: do not know why I want to write but I do
\item[] Let us see this goes to second para or not \sphinxstylestrong{Hide} and \sphinxstyleemphasis{hello}
\end{DUlineblock}

An example of using \DUrole{red}{interpreted text}

\begin{sphinxVerbatim}[commandchars=\\\{\}]
This is a paragraph split across
two lines.
\end{sphinxVerbatim}

\begin{sphinxVerbatim}[commandchars=\\\{\}]
\PYG{p}{\PYGZlt{}}\PYG{n+nt}{html}\PYG{p}{\PYGZgt{}}
    \PYG{p}{\PYGZlt{}}\PYG{n+nt}{head}\PYG{p}{\PYGZgt{}}Hello!\PYG{p}{\PYGZlt{}}\PYG{p}{/}\PYG{n+nt}{head}\PYG{p}{\PYGZgt{}}
    \PYG{p}{\PYGZlt{}}\PYG{n+nt}{body}\PYG{p}{\PYGZgt{}}Hello, world!\PYG{p}{\PYGZlt{}}\PYG{p}{/}\PYG{n+nt}{body}\PYG{p}{\PYGZgt{}}
\PYG{p}{\PYGZlt{}}\PYG{p}{/}\PYG{n+nt}{html}\PYG{p}{\PYGZgt{}}
\end{sphinxVerbatim}

\begin{sphinxVerbatim}[commandchars=\\\{\}]
\PYG{n+nt}{envs\PYGZus{}dirs}\PYG{p}{:}
  \PYG{p+pIndicator}{\PYGZhy{}} \PYG{l+lScalar+lScalarPlain}{\PYGZti{}/my\PYGZhy{}envs}
  \PYG{p+pIndicator}{\PYGZhy{}} \PYG{l+lScalar+lScalarPlain}{/opt/anaconda/envs}
\end{sphinxVerbatim}

\begin{sphinxVerbatim}[commandchars=\\\{\}]
\PYG{p}{:}\PYG{n}{linenos}\PYG{p}{:}

 \PYG{n}{pygments\PYGZus{}style} \PYG{o}{=} \PYG{l+s+s1}{\PYGZsq{}}\PYG{l+s+s1}{sphinx}\PYG{l+s+s1}{\PYGZsq{}}
\end{sphinxVerbatim}


\begin{savenotes}\sphinxattablestart
\centering
\begin{tabular}[t]{|\X{20}{100}|\X{80}{100}|}
\hline
\sphinxstyletheadfamily 
Shapes
&\sphinxstyletheadfamily 
Description
\\
\hline
Square
&
Four sides of equal length, 90 degree angles
\\
\hline
Rectangle
&
Four sides, 90 degree angles
\\
\hline
\end{tabular}
\par
\sphinxattableend\end{savenotes}

\begin{sphinxadmonition}{hint}{Hint:}
This is a admonition of type \sphinxtitleref{hint}.
\end{sphinxadmonition}

\begin{sphinxadmonition}{note}{Note:}
This is a admonition of type \sphinxtitleref{note}.
\end{sphinxadmonition}

\begin{sphinxadmonition}{warning}{Warning:}
This is a admonition of type \sphinxtitleref{warning}.
\end{sphinxadmonition}

\begin{figure}[htbp]
\centering

\noindent\sphinxincludegraphics[width=0.500\linewidth]{{A1}.jpg}
\end{figure}

Paragraph …

\begin{sphinxVerbatim}[commandchars=\\\{\}]
\PYG{n}{Literal} \PYG{n}{Block}
\end{sphinxVerbatim}

Paragraph … :: Literal Block1

A description:

\begin{figure}[htbp]
\centering
\capstart

\noindent\sphinxincludegraphics[scale=0.5]{{A1}.jpg}
\caption{This is the caption of the figure (a simple paragraph).}\label{\detokenize{in_file:id3}}
\begin{sphinxlegend}
The legend consists of all elements after the caption.  In this
case, the legend consists of this paragraph and the following
table:


\begin{savenotes}\sphinxattablestart
\centering
\begin{tabulary}{\linewidth}[t]{|T|T|}
\hline
\sphinxstyletheadfamily 
Symbol
&\sphinxstyletheadfamily 
Meaning
\\
\hline
dddddd
&
Campground
\\
\hline
\noindent\sphinxincludegraphics{{A1}.jpg}
&
Lake
\\
\hline
\noindent\sphinxincludegraphics{{A1}.jpg}
&
Mountain
\\
\hline
\end{tabulary}
\par
\sphinxattableend\end{savenotes}
\end{sphinxlegend}
\end{figure}

\begin{sphinxadmonition}{note}{Note:}
This is a note admonition.
This is the second line of the first paragraph.
\begin{itemize}
\item {} 
The note contains all indented body elements
following.

\item {} 
It includes this bullet list.

\end{itemize}
\end{sphinxadmonition}

\begin{sphinxVerbatim}[commandchars=\\\{\}]
\PYG{n+nt}{envs\PYGZus{}dirs}\PYG{p}{:}
  \PYG{p+pIndicator}{\PYGZhy{}} \PYG{l+lScalar+lScalarPlain}{\PYGZti{}/my\PYGZhy{}envs}
  \PYG{p+pIndicator}{\PYGZhy{}} \PYG{l+lScalar+lScalarPlain}{/opt/anaconda/envs}
\end{sphinxVerbatim}

\begin{DUlineblock}{0em}
\item[] THis is place Raj is putting code
\end{DUlineblock}

\begin{sphinxVerbatim}[commandchars=\\\{\}]
app.controller(\PYGZdq{}myCtrl\PYGZdq{}, function(\PYGZdl{}scope) \PYGZob{}
  \PYGZdl{}scope.firstName = \PYGZdq{}John\PYGZdq{};
  \PYGZdl{}scope.lastName= \PYGZdq{}Doe\PYGZdq{};
\PYGZcb{});
\end{sphinxVerbatim}

\begin{DUlineblock}{0em}
\item[] Lend us a couple of bob till Thursday.Lend us a couple of bob till Thursday.Lend us a couple of bob till Thursday.Lend us a couple of bob till Thursday.Lend us a couple of bob till Thursday.Lend us a couple of bob till Thursday.Lend us a couple of bob till Thursday.Lend us a couple of bob till Thursday.Lend us a couple of bob till Thursday.Lend us a couple of bob till Thursday.
\item[] I’m absolutely skint.
\item[] But I’m expecting a postal order and I can pay you back
\item[]
\begin{DUlineblock}{\DUlineblockindent}
\item[] as soon as it comes.
\end{DUlineblock}
\item[] Love, Ewan.
\item[] Lend us a couple of bob till Thursday.
\end{DUlineblock}

The ‘rm’ command is very dangerous.  If you are logged
in as root and enter

\begin{sphinxVerbatim}[commandchars=\\\{\}]
\PYG{n}{cd} \PYG{o}{/}
\PYG{n}{rm} \PYG{o}{\PYGZhy{}}\PYG{n}{rf} \PYG{o}{*}
\end{sphinxVerbatim}
\noindent
you will erase the entire contents of your file system.


\begin{savenotes}\sphinxattablestart
\centering
\sphinxcapstartof{table}
\sphinxthecaptionisattop
\sphinxcaption{Frozen Delights!}\label{\detokenize{in_file:id4}}
\sphinxaftertopcaption
\begin{tabular}[t]{|\X{15}{55}|\X{10}{55}|\X{30}{55}|}
\hline
\sphinxstyletheadfamily 
Treat
&\sphinxstyletheadfamily 
Quantity
&\sphinxstyletheadfamily 
Description
\\
\hline
Albatross
&
2.99
&
On a stick!
\\
\hline
Crunchy Frog
&
1.49
&
If we took the bones out, it wouldn’t be
crunchy, now would it?
\\
\hline
Gannet Ripple
&
1.99
&
On a stick!
\\
\hline
\end{tabular}
\par
\sphinxattableend\end{savenotes}


\begin{savenotes}\sphinxattablestart
\centering
\sphinxcapstartof{table}
\sphinxthecaptionisattop
\sphinxcaption{Frozen Delights!}\label{\detokenize{in_file:id5}}
\sphinxaftertopcaption
\begin{tabular}[t]{|\X{15}{55}|\X{10}{55}|\X{30}{55}|}
\hline
\sphinxstyletheadfamily 
Treat
&\sphinxstyletheadfamily 
Quantity
&\sphinxstyletheadfamily 
Description
\\
\hline
Albatross
&
2.99
&
On a stick!
\\
\hline
Crunchy Frog
&
1.49
&
If we took the bones out, it wouldn’t be
crunchy, now would it?
\\
\hline
Gannet Ripple
&
1.99
&
On a stick!
\\
\hline
\end{tabular}
\par
\sphinxattableend\end{savenotes}

\begin{sphinxShadowBox}
\sphinxstylesidebartitle{Optional Sidebar Title}
\sphinxstylesidebarsubtitle{Optional Sidebar Subtitle}

Subsequent indented lines comprise
the body of the sidebar, and are
interpreted as body elements.

Subsequent indented lines comprise1q
the body of the sidebar, and are
interpreted as body elements.
\end{sphinxShadowBox}

\begin{DUlineblock}{0em}
\item[] Lend us a couple of bob till Thursday.
\item[] I’m absolutely skint.
\item[] But I’m expecting a postal order and I can pay you back
\item[]
\begin{DUlineblock}{\DUlineblockindent}
\item[] as soon as it comes.
\end{DUlineblock}
\item[] Love, Ewan.
\end{DUlineblock}

This paragraph might be rendered in a custom way.


\chapter{Second Page}
\label{\detokenize{in_file02:second-page}}\label{\detokenize{in_file02::doc}}

\section{Hello ABC}
\label{\detokenize{in_file02:hello-abc}}

\subsection{Chapter 01}
\label{\detokenize{in_file02:chapter-01}}

\chapter{Section 01}
\label{\detokenize{in_file02:section-01}}
Exiting due to level\sphinxhyphen{}4 (SEVERE) system message.

C:Codepython\textgreater{}rst2html.py  \textendash{}stylesheet my.css    in\_file02.rst  out\_file02.html

C:Codepython\textgreater{}rst2html.py  \textendash{}stylesheet my.css    in\_file02.rst  out\_file02.html
in\_file02.rst:28: (SEVERE/4) Title overline \& underline mismatch.
Exiting due to level\sphinxhyphen{}4 (SEVERE) system message.

C:Codepython\textgreater{}rst2html.py  \textendash{}stylesheet my.css    in\_file02.rst  out\_file02.html

C:Codepython\textgreater{}rst2html.py  \textendash{}stylesheet my.css    in\_file02.rst  out\_file02.html
in\_file02.rst:28: (SEVERE/4) Title overline \& underline mismatch.


\chapter{Sectio 02}
\label{\detokenize{in_file02:sectio-02}}
=, for sections

\sphinxhyphen{}, for subsections

\textasciicircum{}, for subsubsections

“, for paragraphs

\sphinxstylestrong{Implict} references, like {\color{red}\bfseries{}\textasciigrave{}Titles are targets, too\textasciigrave{}\_}

Implict references, like {\hyperref[\detokenize{in_file02:hello-abc}]{\sphinxcrossref{Hello ABC}}}

rst2html.py  \textendash{}stylesheet my.css    in\_file02.rst  out\_file02.html

These should be fully qualified CSS color specifiers such as \sphinxcode{\sphinxupquote{\#004B6B}} or
\sphinxcode{\sphinxupquote{\#444}}. The first few items in the list are “global” colors used as defaults
for many of the others; update these to make sweeping changes to the
colorscheme. The more granular settings can be used to override as needed.
\begin{itemize}
\item {} 
\sphinxcode{\sphinxupquote{anchor}}: Foreground color of section anchor links (the ‘paragraph’
symbol that shows up when you mouseover page section headers.)

\item {} 
\sphinxcode{\sphinxupquote{anchor\_hover\_bg}}: Background color of \sphinxcode{\sphinxupquote{anchor}} text.

\item {} 
\sphinxcode{\sphinxupquote{anchor\_hover\_fg}}: Foreground color of section anchor links (as above)
when moused over.

\item {} 
\sphinxcode{\sphinxupquote{body\_text}}: Main content text.

\item {} 
\sphinxcode{\sphinxupquote{code\_highlight}}: Color of highlight when using \sphinxcode{\sphinxupquote{:emphasize\sphinxhyphen{}lines:}} in a code block.

\item {} 
\sphinxcode{\sphinxupquote{footer\_text}}: Footer text (includes links.)

\item {} 
\sphinxcode{\sphinxupquote{footnote\_bg}}: Background of footnote blocks.

\item {} 
\sphinxcode{\sphinxupquote{footnote\_border}}: Border of same.

\item {} 
\sphinxcode{\sphinxupquote{gray\_1}}: Dark gray.

\item {} 
\sphinxcode{\sphinxupquote{gray\_2}}: Light gray.

\item {} 
\sphinxcode{\sphinxupquote{gray\_3}}: Medium gray.

\item {} 
\sphinxcode{\sphinxupquote{link\_hover}}: Body links, hovered.

\item {} 
\sphinxcode{\sphinxupquote{link}}: Non\sphinxhyphen{}hovered body links.

\item {} 
\sphinxcode{\sphinxupquote{narrow\_sidebar\_bg}}: Background of ‘sidebar’ when narrow window forces
it to the bottom of the page.

\item {} 
\sphinxcode{\sphinxupquote{narrow\_sidebar\_fg}}: Text color of same.

\item {} 
\sphinxcode{\sphinxupquote{narrow\_sidebar\_link}}: Link color of same.

\item {} 
\sphinxcode{\sphinxupquote{note\_bg}}: Background of \sphinxcode{\sphinxupquote{.. note::}} blocks.

\item {} 
\sphinxcode{\sphinxupquote{note\_border}}: Border of same.

\item {} 
\sphinxcode{\sphinxupquote{pink\_1}}: Light pink.

\item {} 
\sphinxcode{\sphinxupquote{pink\_2}}: Medium pink.

\item {} 
\sphinxcode{\sphinxupquote{pre\_bg}}: Background of preformatted text blocks (including code
snippets.)

\item {} 
\sphinxcode{\sphinxupquote{relbar\_border}}: Color of border between bar holding \sphinxstyleemphasis{next} and \sphinxstyleemphasis{previous}
links, and the rest of the page content.

\item {} 
\sphinxcode{\sphinxupquote{seealso\_bg}}: Background of \sphinxcode{\sphinxupquote{.. seealso::}} blocks.

\item {} 
\sphinxcode{\sphinxupquote{seealso\_border}}: Border of same.

\item {} 
\sphinxcode{\sphinxupquote{sidebar\_header}}: Sidebar headers.

\item {} 
\sphinxcode{\sphinxupquote{sidebar\_hr}}: Color of sidebar horizontal rule dividers.

\item {} 
\sphinxcode{\sphinxupquote{sidebar\_link}}: Sidebar links (there is no hover variant.) Applies to
both header \& text links.

\item {} 
\sphinxcode{\sphinxupquote{sidebar\_list}}: Foreground color of sidebar list bullets \& unlinked text.

\item {} 
\sphinxcode{\sphinxupquote{sidebar\_link\_underscore}}: Sidebar links’ underline (technically a
bottom\sphinxhyphen{}border).

\item {} 
\sphinxcode{\sphinxupquote{sidebar\_search\_button}}: Background color of the search field’s ‘Go’
button.

\item {} 
\sphinxcode{\sphinxupquote{sidebar\_text}}: Sidebar paragraph text.

\item {} 
\sphinxcode{\sphinxupquote{warn\_bg}}: Background of \sphinxcode{\sphinxupquote{.. warn::}} blocks.

\item {} 
\sphinxcode{\sphinxupquote{warn\_border}}: Border of same.

\end{itemize}


\chapter{Indices and tables}
\label{\detokenize{index:indices-and-tables}}


\renewcommand{\indexname}{Index}
\printindex
\end{document}